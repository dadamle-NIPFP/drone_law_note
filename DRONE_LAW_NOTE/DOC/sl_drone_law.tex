\documentclass[10pt]{beamer}

\usepackage[english]{babel}
\usepackage[utf8]{inputenc}
\usepackage[T1]{fontenc}
\usepackage{pdfpages}

\begin{document}

\section{Need for a new law}

\begin{frame}
  \frametitle{Fundamental shift in technology}
The difference between regular, manned aircraft and civilian UAVs is as
    huge as the difference between the telegraph and the internet. What makes UAVs different from manned aircraft?
  \begin{itemize}
  \item Unmanned: They can be remotely operated, semi-autonomous or completely autonomous.
  \item Size and portability: They can be carried to and flown in places where aeroplanes cannot. 
  \item Accessibility: Upfront capital requirements are a fraction of aeroplanes. They are easy to purchase and fairly easy to operate.
  \item Innovation: UAVs have an enormous number of applications (far greater than manned aircraft), and more are being discovered every day.
  \end{itemize}
\end{frame}

\begin{frame}
  \frametitle{Mismatches / Gaps in existing regulatory framework}
  TO DO
  \begin{itemize}
  \item The legislation and administration (e.g.: DGCA, BCAS etc.) which regulates the aviation sector have been designed expecting a small number of operators, who each own a small number of aircraft. Given the accessibility and utility of UAVs, their numbers are going to be orders of magnitude higher than manned aircraft.
  \item The legislation and administration are also designed assuming that the aircraft fly between a small number of points (i.e. airports) and typically above a certain height.
  \item \textbf{THINK OF MORE}
  \item Existing laws which can impede known drone applications.
    % postage law with drone applications 
    % mapping and survey
    % aerial photography ban (CARs)
  \end{itemize}
\end{frame}

\begin{frame}[breakable]
  \frametitle{Opportunity cost of banning/restricting drones}
\textbf{Proximate projected benefits (approx proportions compared to existing numbers) by leapfrogging:}
  \begin{itemize}
  \item Agriculture --- Savings in input costs, reduction in health expenditure,
  \item Surveying --- Savings in cost of surveying for land admin depts (typically 3 surveys per year), PDNAs after disasters,
  \item Monitoring and evaluation --- Inspecting bridges, dams, railway lines (how many in remote areas?)
  \item Essential items delivery --- Hospitals in remote areas.
  \end{itemize}
\textbf{Mid and long term benefits:}
  \begin{itemize}
  \item New forms of employment,
  \item Decentralisation of monitoring activities: M \& E of big projects can be done by local drone operators (for example),
  \item India is already a significant player in software. Can become a leader in the drone software tech.
  \end{itemize}
  \textbf{Security:}
  \begin{itemize}
  \item Research and development in drones essential for developing anti-drone tech,
  \item Anti-drone tech and reg-tech for drones itself can become a huge market. India can become a leader.
  \end{itemize}
\end{frame}

\section{Guidelines for drafting a new law for civilian UAVs}

\begin{frame}
  \frametitle{Principles for a new UAV law}
  \begin{itemize}
  \item Permitted unless prohibited (E.g.: predetermined no fly zones are better than mandatory permission for every flight);
    % soften a bit
  \item Risk-based regulation-making (CBAs are a must);
  \item Reasonable security considerations;
  \item Lean towards encouraging innovation;
  \item Regulatory sandboxes for testing, dedicated test-sites;
  \item Administration structure should be able to scale with size and needs of industry.
  \end{itemize}
\end{frame}

\end{document}
